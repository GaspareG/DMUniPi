\chapter{Conclusion}

From the analysis obtained in all of the previous phases we have seen that factors like sex, status, education or age don't have a big impact over the credibility of the customers. On the contrary, instead, we have that attributes like the six payment statuses and bill/payment amount means have an high correlation with the credit card defaults. 

\medskip

The customers with the payment status corresponding to the use of revolving credit (that is equal to 0) or with the payment delay for one month (that is equal tu 1), and more in particular in the last two months tracked (September and August), have a very high probability to not being in credit default.

On the other side, customers with the payment status delayed for 2 months or more have an high probability to being in credit default.

\medskip

Using the Model 3 to predict the credit default over the test set provided by \href{https://www.kaggle.com/c/data-mining-20182019-unipi/data}{Kaggle}, we obtain an f1-score of $0.81$, which is higher than obtained in the results of the validation phase.

