\chapter{Association Rules Mining}

\section{Frequent patterns extraction with different parameters}

As we have a lot of categorical attributes with a lot of different values we don't have any frequent pattern with an high support. 

The firsts frequent itemsets appears when we set the minimum support to \textbf{40\%}, in this case we have a total of \textbf{34} itemsets, the first \textbf{10} are:

\begin{enumerate}
  \item support: \textbf{49.55\%}, items: (ps-jun\textbf{0}, ps-may=\textbf{0}).
  \item support: \textbf{48.99\%}, items: (ps-apr=\textbf{0}, ps-may=\textbf{0}).
  \item support: \textbf{48.44\%}, items: (sex=\textbf{female}, default=\textbf{no}).
  \item support: \textbf{48.21\%}, items: (ps-jul=\textbf{0}, ps-jun=\textbf{0}).
  \item support: \textbf{47.69\%}, items: (ps-sep=\textbf{0}, ps-aug=\textbf{0}).
  \item support: \textbf{47.58\%}, items: (ps-aug=\textbf{0}, ps-jul=\textbf{0}).
  \item support: \textbf{45.75\%}, items: (ps-apr=\textbf{0}, ps-jun=\textbf{0}).
  \item support: \textbf{45.42\%}, items: (ps-jul=\textbf{0}, ps-may=\textbf{0}).
  \item support: \textbf{45.31\%}, items: (ps-aug=\textbf{0}, ps-jun=\textbf{0}).
  \item support: \textbf{44.81\%}, items: (ps-may=\textbf{0}, default=\textbf{no}).
\end{enumerate}

In particular we have \textbf{9} itemsets with default equals to \textbf{false} and none with credit default equals to \textbf{yes}.

\smallskip

Choosing a minimum support of \textbf{30\%} results in \textbf{131} itemsets and half of them include the credit default equals to \textbf{no}.

\smallskip

The first frequent itemsets that contains the credit default equals to yes is (sex=\textbf{female}, default=\textbf{yes}) with a support of \textbf{13\%}, unfortunatelly we don't any significant and not naive frequent itemset with a positive credit default so we can only focus on the other ones.

\section{Discussion of the most interesting frequent patterns}

An interesting frequent itemset is the one which have all the six payment statuses equals to \textbf{0} and default to false with a support of \textbf{30.13\%}. This means, due to the properties of the support, that all its subsets (in particular the ones which include default=\textbf{no} and one or more payment status to zero) are frequent itemsets too.

\smallskip

Exluding itemsets which include only payment status attributes or the naive ones, we have also other interesting itemsets, for example:

\begin{enumerate}
  \item support: \textbf{22.44\%}, items: (education=\textbf{university}, sex=\textbf{female}, default=\textbf{no}).
  \item support: \textbf{21.29\%}, items: (age=\textbf{30}, sex=\textbf{female}, default=\textbf{no}).
  \item support: \textbf{20.78\%}, items: (status=\textbf{single}, sex=\textbf{female}, default=\textbf{no}).
\end{enumerate}

We can see that there are some interesting correlation with the sex=\textbf{female} and others frequent attributes (and of course, as said before, with all the payment status equals to \textbf{0}).

\section{Association rules extraction with different values of confidence}

As a consequence of what we have seen before, from practical experiments in the dataset we don't have any significant rule which include a positive credit default, so we again concentrate on the rules with default=\textbf{no}.

\medskip

Trying with some different combination of support/confidence we can find some interesting rules.

Rules for which it is default=\textbf{no}:

\begin{itemize}
  \item (sex=\textbf{female}, at least \textbf{4} different payment statuses equals to \textbf{0}):
  
  \tab support of \textbf{20\%} and confidence of \textbf{90\%}.

  \item (sex=\textbf{female}, ba\_mean=\textbf{50000} NTD, at least \textbf{4} different payment statuses equals to \textbf{0}):
  
  \tab support of \textbf{2\%} and confidence of \textbf{95\%}.

\end{itemize}

Rules that doesn't include credit default attribute:

\begin{itemize}
  \item (ps-sep=\textbf{0}, at least two payment statuses equals to \textbf{0} (from april to july)) $\rightarrow$ (ps-aug=\textbf{0}):
  \tab support of \textbf{30\%} and confidence of \textbf{99\%}.
\end{itemize}

\section{Discussion of the most interesting rules}

We can see again that the most interesting attribute values in the rules are sex=\textbf{female} and payment status=\textbf{0} for all the months. There are a lot of rules with both high accuracy and support which consists in combinations of these two kind of attributes. 

\smallskip

A couple of rules that don't involve payment amounts:

\begin{itemize}
  \item (education=\textbf{graduate school}, sex=\textbf{female}) $\rightarrow$ (default=\textbf{no}):
  
    \tab accuracy of \textbf{81.68\%} and support of \textbf{16.87\%}.
  \item (age=\textbf{30-39}, sex=\textbf{female}) $\rightarrow$ (default=\textbf{no}):
  
    \tab accuracy of \textbf{80.71\%} and support of \textbf{21.29\%}.
\end{itemize}

\section{Use the most meaningful rules to replace missing values}

The dataset doesn't contains a lot of missing values related to the most interesting rules, so we decide to simulate the substition, in this way we can evalute the accuracy of these rules.

\begin{itemize}
  \item (age=\textbf{30-39}, status=\textbf{married}, education=\textbf{university}) $\rightarrow$ (sex=\textbf{female}):
  
    \tab accuracy of \textbf{73.71\%} and support of \textbf{3.36\%}.
  \item (payment status equals to \textbf{0} from april to august) $\rightarrow$ (ps\_sep=\textbf{0}):
  
    \tab accuracy of \textbf{91.85\%} and support of \textbf{36.59\%}.
  \item (payment status equals to \textbf{0} from april to july) $\rightarrow$ (ps\_aug=\textbf{0}, ps\_sep=\textbf{0}):
  
    \tab accuracy of \textbf{84.18\%} and support of \textbf{39.92\%}.

  \item (payment status equals to \textbf{0} from april to june) $\rightarrow$ (ps\_jul=\textbf{0}, ps\_aug=\textbf{0}, ps\_sep=\textbf{0}):
  
    \tab accuracy of \textbf{76.43\%} and support of \textbf{43.97\%}.

  \item (payment status equals to \textbf{0} from april to may) $\rightarrow$ (ps\_jun=\textbf{0}, ps\_jul=\textbf{0}, ps\_aug=\textbf{0}, ps\_sep=\textbf{0}):
  
    \tab accuracy of \textbf{68.60\%} and support of \textbf{48.99\%}.

\end{itemize}

\clearpage

\section{Use the most meaningful rules to predict credit card defaults}

We first decide to test some rules over only the attributes relative to the payment status:

\begin{itemize}
  \item (all \textbf{6} payment statuses equals to \textbf{0}) $\rightarrow$ (default=\textbf{no}):
  
    \tab accuracy of \textbf{89.66\%} and support of \textbf{33.61\%}.
  \item (at least \textbf{5} payment statuses equals to \textbf{0}) $\rightarrow$ (default=\textbf{no}):
  
    \tab accuracy of \textbf{85.88\%} and support of \textbf{43.18\%}.
  \item (at least \textbf{4} payment statuses equals to \textbf{0}) $\rightarrow$ (default=\textbf{no}):
  
    \tab accuracy of \textbf{83.14\%} and support of \textbf{50.66\%}.
  \item (at least \textbf{3} payment statuses equals to \textbf{0}) $\rightarrow$ (default=\textbf{no}):
  
    \tab accuracy of \textbf{80.90\%} and support of \textbf{57.37\%}.
  \item (at least \textbf{2} payment statuses equals to \textbf{0}) $\rightarrow$ (default=\textbf{no}):
  
    \tab accuracy of \textbf{78.98\%} and support of \textbf{64.93\%}.
  \item (at least \textbf{1} payment statuses equals to \textbf{0}) $\rightarrow$ (default=\textbf{no}):
  
    \tab accuracy of \textbf{78.61\%} and support of \textbf{69.88\%}.

\end{itemize} 

With an high accuracy and support we can conclude that these are good rules to predict the negative credit default.

\smallskip

By introducing the condition sex=female in the previous rules doesn't change a lot the results, for example:

\begin{itemize}
  \item (sex=\textbf{female}, all \textbf{6} payment statuses equals to \textbf{0}) $\rightarrow$ (default=\textbf{no}):
  
    \tab precision \textbf{90.22\%}, support \textbf{20.86\%}.
  \item (sex=\textbf{female}, at least \textbf{5} payment statuses equals to \textbf{0}) $\rightarrow$ (default=\textbf{no}):
  
    \tab precision \textbf{88.80\%}, support \textbf{26.58\%}.
\end{itemize}

In particular we have a big decrease in support with a small increase in precision.
