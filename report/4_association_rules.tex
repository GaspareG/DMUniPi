\chapter{Association Rules Mining}

\section{Frequent patterns extraction with different parameters}

As we have a lot of categorical attributes with a lot of different values we don't have any frequent pattern with an high support. 

The firsts frequent itemsets appears when we set the minimum support to $40\%$, in this case we have a total of 34 itemsets, the first 10 are:

\begin{enumerate}
  \item support: 49.55\%, items: (ps-jun=0, ps-may=0)
  \item support: 48.99\%, items: (ps-apr=0, ps-may=0)
  \item support: 48.44\%, items: (sex=female, default=false)
  \item support: 48.21\%, items: (ps-jul=0, ps-jun=0)
  \item support: 47.69\%, items: (ps-sep=0, ps-aug=0)
  \item support: 47.58\%, items: (ps-aug=0, ps-jul=0)
  \item support: 45.75\%, items: (ps-apr=0, ps-jun=0)
  \item support: 45.42\%, items: (ps-jul=0, ps-may=0)
  \item support: 45.31\%, items: (ps-aug=0, ps-jun=0)
  \item support: 44.81\%, items: (ps-may=0, default = false)
\end{enumerate}

In particular we have $9$ itemsets with default equals to false and none with credit default equals to false.

\smallskip

Choosing a minimum support of $30\%$ results in $131$ itemsets and half of them include the credit default equals to false.

\smallskip

The first frequent itemsets that contains the credit default equals to true is (sex=female, default=true) with a support of $13\%$, unfortunatelly we don't any significant and not naive frequen itemset with a positive credit default so we can only focus on the other ones.

\section{Discussion of the most interesting frequent patterns}

An interesting frequent itemset is the one which have all the six payment statuses equals to $0$ and default to false with a support of $30.13\%$. This means, due to the properties of the support, that all its subsets (in particular the ones which include default to false and one or more payment status to zero) are frequent itemsets too.

\smallskip

Exluding itemsets which include only payment status attributes or the naive ones, we have also other interesting itemsets, for example:

\begin{enumerate}
  \item support: 22.44\%, items: (education=university, sex=female, default=false)
  \item support: 21.29\%, items: (age=30, sex=female, default=false)
  \item support: 20.78\%, items: (status=single, sex=female, default=false)
\end{enumerate}

We can see that there are some interesting correlation with the sex=female and others frequent attibutes (and of course, as said before, with all the payment status equals to 0).

\section{Association rules extraction with different values of confidence}

As a consequence of what we have seen before, from practical experiments in the dataset we don't have any significant rule which include a positive credit default, so we again concentrate on the rules with default=false.

\medskip

Trying with some different combination of support/confidence we can find some interesting rules.

Rules for which it is default=false:

\begin{itemize}
  \item support of $20\%$ and confidence of $90\%$: 
  sex=female and at least $4$ different payment statuses equals to 0.

  \item support of $2\%$ and confidence of $95\%$: 
  sex=female, average bill amount=$50000$ NTD and at least $4$ different payment statuses equals to 0.

\end{itemize}

Rules that doesn't include credit default attribute:

\begin{itemize}
  \item support of $30\%$ and confidence of $99\%$: 
  ps-sep=0 and at least two payment statuses equals to 0 (from april to july) $\rightarrow$ ps-aug=0.
\end{itemize}

\section{Discussion of the most interesting rules}

We can see again that the most interesting attribute values in the rules are sex=female and payment status=0 for all the months. 

\smallskip

There are a lot of rules with both high accuracy and support of the form 

\smallskip

A couple of rules that don't involve payment amounts:

\begin{itemize}
  \item education=graduate school, sex=female $\rightarrow$ credit default = false:
  
    \tab accuracy $81.68\%$, support $16.87\%$
  \item age=30-39, sex=female $\rightarrow$ credit default = false:
  
    \tab accuracy $80.71\%$, support $21.29\%$
\end{itemize}

\section{Use the most meaningful rules to replace missing values}

The dataset doesn't contains a lot of missing values related to the most interesting rules, so we decide to simulate the substition, in this way we can evalute the accuracy of these rules.

\begin{itemize}
  \item (age=30-39, status=married, educatio=university) $\rightarrow$ sex=female:
  
    \tab accuracy $73.71\%$, support $3.36\%$
  \item (payment status equals to 0 from march to august) $\rightarrow$ ps\_sep=0:
  
    \tab accuracy $91.85\%$, support $36.59\%$
  \item (payment status equals to 0 from march to july) $\rightarrow$ (ps\_aug=0, ps\_sep=0):
  
    \tab accuracy $84.18\%$, support $39.92\%$

  \item (payment status equals to 0 from march to june) $\rightarrow$ (ps\_jul=0, ps\_aug=0, ps\_sep=0):
  
    \tab accuracy $76.43\%$, support $43.97\%$

\end{itemize}

\clearpage

\section{Use the most meaningful rules to predict credit card defaults}

We first decide to test some rules over only the attributes relative to the payment status:

\begin{itemize}
  \item (all 6 payment statuses equals to 0) $\rightarrow$ credit default = false:
  
    \tab accuracy $89.66\%$, support $33.61\%$
  \item (at least 5 payment statuses equals to 0) $\rightarrow$ credit default = false:
  
    \tab accuracy $85.88\%$, support $43.18\%$
  \item (at least 4 payment statuses equals to 0) $\rightarrow$ credit default = false:
  
    \tab accuracy $83.14\%$, support $50.66\%$
  \item (at least 3 payment statuses equals to 0) $\rightarrow$ credit default = false:
  
    \tab accuracy $80.90\%$, support $57.37\%$
  \item (at least 2 payment statuses equals to 0) $\rightarrow$ credit default = false:
  
    \tab accuracy $78.98\%$, support $64.93\%$
  \item (at least 1 payment statuses equals to 0) $\rightarrow$ credit default = false:
  
    \tab accuracy $78.61\%$, support $69.88\%$

\end{itemize} 

With an high accuracy and support we can conclude that these are good rules to predict the negative credit default.

\smallskip

By introducing the condition sex=female in the previous rules doesn't change a lot the results, for example:

\begin{itemize}
  \item sex=female,  (all 6 payment statuses equals to 0) $\rightarrow$ credit default = false:
  
    \tab precision $90.22\%$, support $20.86\%$
  \item sex=female, (at least 5 payment statuses equals to 0) $\rightarrow$ credit default = false:
  
    \tab precision $88.80\%$, support $26.58\%$
\end{itemize}

In particular we have a big decrease in support with a small increase for the precision.
